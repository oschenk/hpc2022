\documentclass[unicode,11pt,a4paper,oneside,numbers=endperiod,openany]{scrartcl}

\usepackage{amsmath}
\usepackage{ifthen}
\usepackage[utf8]{inputenc}
\usepackage{graphics}
\usepackage{graphicx}
\usepackage{hyperref}

\pagestyle{plain}
\voffset -5mm
\oddsidemargin  0mm
\evensidemargin -11mm
\marginparwidth 2cm
\marginparsep 0pt
\topmargin 0mm
\headheight 0pt
\headsep 0pt
\topskip 0pt        
\textheight 255mm
\textwidth 165mm

\newcommand{\duedate} {}
\newcommand{\setduedate}[1]{%
\renewcommand\duedate {Due date:~ #1}}
\newcommand\isassignment {false}
\newcommand{\setassignment}{\renewcommand\isassignment {true}}
\newcommand{\ifassignment}[1]{\ifthenelse{\boolean{\isassignment}}{#1}{}}
\newcommand{\ifnotassignment}[1]{\ifthenelse{\boolean{\isassignment}}{}{#1}}

\newcommand{\assignmentpolicy}{
\begin{table}[h]
\begin{center}
\scalebox{0.8} {%
\begin{tabular}{|p{0.02cm}p{16cm}|}
\hline
&\\
\multicolumn{2}{|c|}{\Large\textbf{HPC  2021 ---  Submission Instructions}}\\
\multicolumn{2}{|c|}{\large\textbf{(Please, notice that following instructions are mandatory: }}\\
\multicolumn{2}{|c|}{\large\textbf{submissions that don't comply with, won't be considered)}}\\
&\\
\textbullet & Assignments must be submitted to \href{https://www.icorsi.ch/course/view.php?id=12615}{iCorsi} (i.e. in electronic format).\\
\textbullet & Provide both executable package and sources (e.g. C/C++ files, Matlab). 
If you are using libraries, please add them in the file. Sources must be organized in directories called:\\
\multicolumn{2}{|c|}{\textit{Project\_number\_lastname\_firstname}}\\
& and  the  file must be called:\\
\multicolumn{2}{|c|}{\textit{project\_number\_lastname\_firstname.zip}}\\
\multicolumn{2}{|c|}{\textit{project\_number\_lastname\_firstname.pdf}}\\
\textbullet &  The TAs will grade your project by reviewing your project write-up, and looking at the implementation 
                 you attempted, and benchmarking your code's performance.\\

\textbullet & You are allowed to discuss all questions with anyone you like; however: (i) your submission must list anyone you discussed problems with and (ii) you must write up your submission independently.\\
\hline
\end{tabular}
}
\end{center}
\end{table}
}
\newcommand{\punkte}[1]{\hspace{1ex}\emph{\mdseries\hfill(#1~\ifcase#1{Points}\or{Points}\else{Points}\fi)}}


\newcommand\serieheader[6]{
\thispagestyle{empty}%
\begin{flushleft}
\includegraphics[width=0.4\textwidth]{usi_inf.png}
\end{flushleft}
  \noindent%
  {\large\ignorespaces{\textbf{#1}}\hspace{\fill}\ignorespaces{ \textbf{#2}}}\\ \\%
  {\large\ignorespaces #3 \hspace{\fill}\ignorespaces #4}\\
  \noindent%
  \bigskip
  \hrule\par\bigskip\noindent%
  \bigskip {\ignorespaces {\Large{\textbf{#5}}}
  \hspace{\fill}\ignorespaces \large \ifthenelse{\boolean{\isassignment}}{\duedate}{#6}}
  \hrule\par\bigskip\noindent%  \linebreak
 }

\makeatletter
\def\enumerateMod{\ifnum \@enumdepth >3 \@toodeep\else
      \advance\@enumdepth \@ne
      \edef\@enumctr{enum\romannumeral\the\@enumdepth}\list
      {\csname label\@enumctr\endcsname}{\usecounter
        {\@enumctr}%%%? the following differs from "enumerate"
	\topsep0pt%
	\partopsep0pt%
	\itemsep0pt%
	\def\makelabel##1{\hss\llap{##1}}}\fi}
\let\endenumerateMod =\endlist
\makeatother




\usepackage{textcomp}





\begin{document}


\setassignment
\setduedate{21.12.2022, 23:59}

\serieheader{High-Performance Computing}{2022}{Student: FULL NAME}{Discussed with: FULL NAME}{Solution for Project 7}{}
\newline

\assignmentpolicy

% -------------------------------------------------------------------------- %
% -------------------------------------------------------------------------- %
% --- Exercise 1 ----------------------------------------------------------- %
% -------------------------------------------------------------------------- %
% -------------------------------------------------------------------------- %

\section{Parallel Space Solution of a nonlinear PDE using MPI [in total 35 points]}

\subsection{Initialize and finalize MPI [5 Points]}

\subsection{Create a Cartesian topology [5 Points]}

\subsection{Extend the linear algebra functions [5 Points]}

\subsection{Exchange ghost cells [10 Points]}

\subsection{Scaling experiments [10 Points]}


 % -------------------------------------------------------------------------- %
% -------------------------------------------------------------------------- %
% --- Exercise 2 ----------------------------------------------------------- %
% -------------------------------------------------------------------------- %
% -------------------------------------------------------------------------- %

\section{Python for High-Performance Computing (HPC) [in total 50 points]}

\subsection{Sum of ranks: MPI collectives [5 Points]}

\subsection{Domain decomposition: Create a Cartesian topology [5 Points]}

\subsection{Exchange rank with neighbours [5 Points]}

\subsection{Change linear algebra functions [5 Points]}

\subsection{Exchange ghost cells [5 Points]}

\subsection{Scaling experiments [5 Points]}

\subsection{A self-scheduling example: Parallel Mandelbrot [20 Points]}


\section{Task:  Quality of the Report [15 Points]}
Each project will have 100 points (out of  which 15 points will be given to the general quality of the written report).


\section*{Additional notes and submission details}
Submit the source code files (together with your used \texttt{Makefile}) in
an archive file (tar, zip, etc.), and summarize your results and the
observations for all exercises by writing an extended Latex report.
Use the Latex template from the webpage and upload the Latex summary
as a PDF to \href{https://www.icorsi.ch/course/view.php?id=14652}{iCorsi}.

\begin{itemize}
	\item Your submission should be a gzipped tar archive, formatted like project\_number\_lastname\_firstname.zip or project\_number\_lastname\_firstname.tgz. 
	It should contain:
	\begin{itemize}
		\item all the source codes of your solutions;
		\item your write-up with your name  project\_number\_lastname\_firstname.pdf.
	\end{itemize}
	\item Submit your .tgz through Icorsi.
\end{itemize}

\end{document}
